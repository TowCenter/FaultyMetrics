Dear Readers,
We are proud to present to you the first article of research from the Tow Center
for Digital Journalism, which is launching this fall at the Columbia University
Graduate School of Journalism.
In this report, our researchers dig deeply into the many traffic metrics that
confront online news sites, and in the process, they provide insight into how that
cacophony of numbers affects everything from advertising models to editorial
decisions.
You’ll notice that the authors also take an unusual additional step at the end of
the report: They offer recommendations on how and why the Tow Center could
play a role in explaining these metrics and helping the industry deal with them.
We think it is most appropriate, given the nature of the Internet, that we openly
solicit your thoughts about how to shape some of the research goals of the Tow
Center. And so we invite you to read this report, provide feedback on the
findings, and give us your ideas about what research areas would be most
valuable for the Center to pursue. You can leave comments on the accompanying
article on the Columbia Journalism Review, http://www.cjr.org/reports/
traffic_jam.php, or write us at our email addresses below.
We also would like to thank Mary Graham, a member of the Journalism School’s
Board of Visitors, whose generosity enabled the Tow Center to produce this
report.
Thank you, and we look forward to hearing from you.
Sincerely,
Bill Grueskin Emily Bell
Dean of Academic Affairs Director, Tow Center for Digital Journalism
Columbia School of Journalism Columbia School of Journalism
bgrueskin@columbia.edu eb2740@columbia.edu
Executive Summary
The New York Times is one of the most popular news destinations on the
Internet. Its online audience has been growing steadily for years, but over
the first half of 2010 the number of monthly visitors to Times‐owned sites
surged — from 53 million to 72 million people, according to comScore,
one of the leading firms tracking Web usage.
Why the sudden boost? The surge was a methodological anomaly:
comScore decided to change the way it counts online users. Both
comScore and its biggest rival, Nielsen NetRatings, have been revamping
their secret formulas to bring their audience tallies closer to what online
media outlets claim. But the two firms don’t often agree with each other,
either. In May comScore gave Washingtonpost.com an audience of 17
million ``unique visitors,'' while Nielsen recorded fewer than 10 million.
Their calculations of Yahoo’s audience differed by 34 million people,
roughly the population of Canada.
Media measurement has never been an exact science. But even by its
imperfect standards, Internet audience estimates vary to an astonishing
degree, depending on who does the counting and what methodology is
applied. These swings are especially challenging for smaller, online‐only
properties trying to get on the radar of major advertisers.
Consider The Daily Beast, the news‐and‐opinion outlet edited by Tina
Brown and owned by Interactive Corp. As reported in an LA Times profile
early this year, Nielsen put the site’s audience at 1 million for October of
2009; the same month comScore counted 2.2 million visitors, more than
twice as many. Meanwhile, according to the site’s own servers, close to 4
million different users were reading the Beast each month.
Comparisons like these are far from unusual. A striking paradox exists in
the world of Internet measurement, whose math shapes the fortunes of
news organizations both new and old: What is supposedly the most
measurable medium in history is beset by a frightening tangle of
incompatible standards and contradictory results.
This messy landscape poses a stark contrast to traditional media.
Journalists working in newspapers, magazines, radio, and television can
rely on a dominant ``currency'' (though always an imperfect one) to
measure audiences, close advertising deals, and assess the competition.
This report explores the industry of Internet measurement and its impact
on news organizations working online. It investigates this landscape
through a combination of documentary research and interviews with
measurement companies, trade groups, advertising agencies, media
scholars, and journalists from national newspapers, regional papers, and
online‐only news ventures. Principal findings include the following:
* Major online news outlets routinely subscribe to multiple, incompatible,
and quite expensive sources of audience measurement, picking and
choosing data to tell a compelling story to advertisers. Smaller ventures
rely mainly on Google Analytics.
* Uncertainty about audience measurement hinders online ad spending,
with buyers and sellers of media favoring incompatible metrics. (A 2009
study by McKinsey & Co., commissioned by the Internet Advertising
Bureau, echoed this finding.)
* Uncertainty about audience measurement impedes editorial decisionmaking,
with editors unsure of which readers favor what coverage.
Editors still choose among costly projects by instinct; as one reported,
``You have more data, but itʹs conflicting.''
* The media‐planning dynamic is inverted online: marketers allocate more
resources to optimizing campaigns as they run, rather than to planning
them beforehand. Even brand advertisers have adopted the thinking of
the less‐fashionable world of direct marketing.
* Advertising technologies used online, such as behavioral targeting, tend
to erode the value of a news outlet’s audience profile. Increasingly, the
decisive information resides not with the publisher but in the databases of
intermediaries such as ad networks or profile brokers.
* As a result, despite widespread calls for a common currency, the online
ad industry does not depend on having a single measurement standard
like Nielsen’s TV ratings. In contrast to the world of television or
magazines, space for advertising is not a scarce resource on the Internet,
and online marketers don’t rely on ratings in the same way to purchase
media or to evaluate their campaigns.
Thus the chaos of competing metrics online does not represent the failure
of the Internet’s promise as the most ``accountable'' medium, but, in some
ways, its realization. The global network generates far more natural data
about audiences than any prior mass medium; at the same time it
diminishes the need to anoint a single, arbitrary standard for the sake of
agreement. It turns out that accountability is a messy business.
This report identifies two routes which may bring a measure of consensus
to this fractured landscape: A merger between the two top online ratings
firms, comScore and Nielsen, or the emergence of Google Analytics as an
accepted standard. Neither of these paths is assured, however, and neither
would achieve the ``clarity'' of traditional media currencies — a clarity
that results more from a lack of data than from good data.
In this environment, the report identifies three future avenues of inquiry
in pursuit of the Tow Center’s mission to foster viable and effective digital
journalism:
First, educating journalists to navigate the chaos of data about online audiences,
and in particular about journalism on the Internet.
Second, developing resources to help journalists understand the impact of their
work, beyond counting ``eyeballs'' — thoughtful measures of how news travels
and what effect it has in a networked information economy.
Third, producing much‐needed research on emerging business models for
professional journalism, in order to understand how high‐quality reporting can
thrive when old media economics no longer apply.
I. Fractured Media Metrics: The Lack of an Online ``Currency''
A striking contradiction exists at the center of the confusing world of
Internet metrics: What is by all accounts the most precisely measurable
medium in history, in which every act of reading, watching or listening is
a discrete, recorded event, is beset by a frightening tangle of incompatible
standards for gauging traffic.
Every new medium has endured a period of statistical upheaval. Without
exception, though, major ad‐supported media platforms — newspapers,
magazines, TV networks, radio stations —have settled on one dominant,
third‐party standard for counting audiences. In contrast, the online
landscape today is, if anything, more fractured and confusing than in the
Internetʹs earliest days as a popular medium, still characterized by basic
disagreements over not just how but what to measure. This cacophony
persists despite the clear maturation of the online advertising industry,
which according to Forrester Research will claim \$29 billion in the United
States in 2010, or 13 percent of total ad spending (though search‐engine
marketing accounts for more than half of the online share).
Among online news outlets of various stripes, the perception of a chaos of
competing metrics seems to be universal. This is a troubling issue for these
publishers, editors, and reporters. As they seek to perform powerful
journalism with a wide impact, they are befuddled by contradictory data
sets that fail to capture how their stories are being distributed or read, and
what sort of impact they are having on their audience or on the
institutions they cover. Furthermore, as they seek to build sustainable
business models in the online economy, they have a hard time finding the
reliable, consistent data that allow other industries to grow and thrive.
Several industry groups, representing publishers as well as advertisers
and the measurement firms themselves, have launched initiatives that aim
to bring clarity and consensus to the online measurement landscape. To
understand what online news ventures can or even should hope for in
such efforts, and what kind of contribution the Tow Center can make,
requires first understanding why a measurement currency has not
emerged online and whether one is likely to.
The key question, in other words, is whether the continuing disagreement
over online measurement standards on the Internet is evidence of a young
medium, or of a fundamentally different one. Will an agreed‐upon
currency emerge for counting audiences on the Internet? Why hasnʹt one
taken hold thus far? Answering these questions begins with a closer look
at measurement currencies in traditional media.
Approaches to traditional media measurement
Every major platform for news, as for media more broadly, relies heavily
on third‐party measurement firms. This is especially true in the corners of
journalism that count on advertising as a main revenue source. From the
perspective of a publisher or broadcaster of news, media measurement —
which typically means audience measurement, even when audiences
arenʹt being directly polled — fulfills three distinct but overlapping needs:
\begin{itemize}
\item Understanding audiences, for editorial as well as commercial
\end{itemize}
development.
\begin{itemize}
\item Evaluating competitors.
\item Selling ad space. This includes marketing audiences to advertisers
\end{itemize}
as well as setting ad rates and closing deals.
In each case, the most basic role of media measurement is to achieve a
consensus on the number of people reading, watching, or listening to a
particular news outlet — and to the ads it carries. Has readership
increased since the redesign? Did last week’s feature win viewers over?
Can we command a premium with advertisers, based on our demographic
profile? All of these questions revolve around measuring audiences.
Source: authors’ research
The \$19 billion U.S. market research industry^{\href{#endnotes}{1}} harbors many media
tracking firms, offering a dizzying array of products based on various
methodologies and data sources. However, among firms that measure
media audiences, two broad approaches prevail. Panel‐based measures
such as Nielsen’s TV ratings operate by tracking media usage within a
small, carefully maintained panel of media users and extrapolating their
habits to the broader population. Census‐based measures, so‐called
because they purport to reflect the entire universe of media users rather
than just a sample, are possible only where distribution offers some clue
about that the size of universe — for instance, in records of the number of
copies of a newspaper printed and sold each day. (A more accurate split
might be between estimates derived directly from audiences and those
that begin with media producers.)
1 According to the 2010 Gale Encyclopedia of American Industries.
The role of media measurement ``currencies''
Historically the advertising market has been the basis for powerful,
decades‐long monopolies in audience measurement: Firms such as the
Nielsen Company, Arbitron, and the Audit Bureau of Circulations
provide the ``currency'' that buyers and sellers of media use to set ad rates
based on the size and quality of the audience being reached.^{\href{#endnotes}{2}} This is true
despite the fact that media measurement has been plagued by controversy
over methodology and business practices.
Consider Nielsen’s eponymous TV ratings, the undisputed currency of
both broadcast and cable television. As early as 1963, doubts about the
accuracy and fairness of broadcast ratings led to a series of Congressional
hearings. One contemporary account summed up the findings this way:
``The hearings suggested that the illusion of exact accuracy was
necessary to the ratings industry in order to heighten the
confidence of their clients in the validity of the data they sell. This
myth was sustained by the practice of reporting audience ratings
down to the decimal point, even when the sampling tolerances
ranged over several percentage points. It was reinforced by keeping
as a closely guarded secret the elaborate weighting procedures
which were used to translate interviews into published projections
of audience size. It was manifested in the monolithic self‐assurance
with which the statistical uncertainties of survey data were
transformed into beautiful, solid, clean‐looking bar charts.''^{\href{#endnotes}{3}}
Still, Nielsen’s weekly ratings have enjoyed a half‐century reign despite
any number of critiques leveled against the system since then: that selfreported
viewer ``diaries'' are unreliable; that the household panel has
been too small; that the panel undercounts out‐of‐home viewers, for
instance in college dorms; that the panel undercounts minorities; and most
gravely, that the panel is not a truly random statistical sample of US
2 This discussion of ``currency'' metrics draws on M.R. Sales & M. Gluck, The Future
of Television: Advertising, Technology and the Pursuit of Audiences, The Norman
Lear Center at the Annenberg School for Communication, September 2008.
3 L. Bogart, ``Is it Time to Discard the Audience Concept?,'' Journal of Marketing 30
(1), January 1966.
households.^{\href{#endnotes}{4}} Nielsen’s many methodological tweaks over the decades —
such as increasing the size of its panel and deploying automated
measurement technologies — suggest that these critiques have carried
some weight.
It is a long‐appreciated irony of media measurement that accuracy matters
less than consensus. A media executive may have little faith in the
formula used to infer the radio choices of millions of commuters, or in the
``pass‐along'' multiple that transforms a small newspaper circulation into
a much wider assumed readership; these doubts don’t matter much as
long as no competitor is seen to benefit. As an ABC executive put it in a
1992 PBS documentary about flaws in Nielsenʹs methodology,
``Everybodyʹs dealing off the same deck.''^{\href{#endnotes}{5}}
The history of radio, television, and print strongly supports the conclusion
that buyers and sellers of media invariably anoint a single, third‐party
``currency'' for counting audiences. This may be a messy process — for
instance, Arbitron originally formed in 1949 (as the American Research
Bureau) to track television viewing, and competed with Nielsen for
decades before finally conceding defeat in 1993. Nevertheless, Nielsen led
its rival throughout that period, and only became stronger as fewer and
fewer clients were willing to subscribe to more than one ratings firm.
Similarly, Arbitron has faced a number of challengers in ranking radio
audiences (including Nielsen, which launched the first radio ratings
service in 1942 and re‐entered the business last year) but has dominated
the industry for four decades. Each firm has also weathered profound
change in its industry, for instance the growth of cable TV in the 1980s,
and the dramatic consolidation of radio after 1996.
At first glance, print media seem to offer an exception, with two
measurement firms maintaining competing currencies. However, each
firm dominates a different segment of the print landscape: The Audit
Bureau of Circulations, founded in 1914, is the audience currency among
newspapers while BPA Worldwide, founded in 1931 as Controlled
Circulation Audit, has much deeper support in the magazine industry and
overseas.
4 For recent examples see M. Schneider, ``Fox Wants Answers from Nielsen,'' Variety,
18 May 2009; and A. Segal, ``Nielsen Ratings: An Inaccurate Truth,'' Cornell Daily
Sun, 26 April 2007.
5 ``Can You Believe TV Ratings,'' Nova, PBS, 1992.
How measurement monopolies survive
Clearly, incumbent media measurement standards benefit from network
effects. The more widely a currency such as Nielsen’s ratings system is
used to negotiate TV ad deals, the more necessary it is for any TV network
or ad agency to subscribe, further cementing Nielsen’s status as an
industry standard. (This competitive advantage is enhanced by barriers to
entry, such as the high fixed costs of establishing a viable panel without a
base of clients already in place.) Given such a feedback loop, even a small
lead by one measurement firm will – all other things being equal –
eventually lead to outright dominance. This pattern suggests that a
divided measurement markets are inherently unstable, and thus that a
single measurement currency will emerge online as it has in other
platforms.
The reality is messier than this. Two crucial features of existing
measurement monopolies deserve close attention. First, and
paradoxically, a single currency appears to be most dominant precisely in
the broadcast platforms — television and radio — where natural data
about audiences are absent. Conversely in print, where more complete,
census‐like information about the size (and to some extent the quality) of
the reading universe has always been available from subscription and
newsstand figures, the measurement landscape remains more fractured.
Upon reflection, the paradox disappears. To compare separately audited
(or even unaudited) newspaper circulation figures is less than ideal. But to
compare reach estimates projected from separate audience panels makes
almost no sense, since the panels may be biased in different ways. (This
has become painfully clear in the disagreement between the two online
panels, discussed in the next section.)
Second, the role of advertisers in anointing currency metrics cannot be
ignored. Advertisers and ad agencies drove the formation of the
circulation auditors ABC and BPA; funding for the two nonprofit auditors
comes from dues paid by advertisers, agencies, and media companies, but
agencies and advertisers dominate boards of both organizations.
(Newspaper publishers are wary of ABC’s online auditing proposals
partly because advertisers have so much influence over the group.)
Likewise, both broadcasters and advertising agencies pay to subscribe to
Nielsen’s television ratings. The firm’s pricing is opaque, but full
subscriptions for commercial clients run to tens of millions of dollars per
year. Media producers have a clear interest in understanding and
comparing their own audiences. But it remains an open question whether
a single measurement currency would emerge in the absence of pressure
from advertisers and agencies.
Finally, established measurement firms work hard to reinforce network
effects and protect their monopoly. A report from USC Annenberg’s Lear
Center points to Nielsen’s ``carefully staggered annual contracts,'' which
make it extremely difficult for a challenger to win over a critical mass of
clients.^{\href{#endnotes}{6}} As Arbitron’s CEO told the New York Times about its decision not
to get back into TV ratings in the 1990s, ``We looked at this and saw that
thereʹs a long history of people taking runs at the incumbent. \ldots  But
thereʹs no halfway here. If we were to go after Nielsen, it would be war,
and at the end of the day there would be one person standing. And
believe me, there are skeletons littering the trail.''^{\href{#endnotes}{7}}
Why no currency for online measurement?
As noted at the outset, the online measurement landscape remains
extremely fractured. Two major firms, Nielsen NetRatings and comScore,
are vying to become the industry standard in panel‐based audience
measurement; several smaller firms also compete in this arena. But as the
next section explores in detail, publishers who subscribe to one (or both)
of these typically also employ census‐based audience measurement,
which in the online world means analyzing server‐side records of audio,
video, and text pages served out over the Internet. (Many companies offer
this kind of analysis; two leading services are Omniture and Google
Analytics.) Meanwhile a third kind of measurement firm — Hitwise is an
example — measures audiences using data aggregated from internet
service providers (ISPs).^{\href{#endnotes}{8}} The closest analogy in the offline world might be
polling retailers about who is buying what magazines. Finally any number
of startups are attempting to measure audiences and activity on new
platforms such as mobile devices and social networks.
6 The Future of Television.
7 J. Gertner, ``Our Ratings, Ourselves,'' New York Times, 10 April 1995; cited in The
Future of Television.
8 For a good review of contenders see ``Sizing Up Online Audience Measurement
Services'' from the Newspaper Association of America, at
http://www.naa.org/Resources/Articles/Digital-Edge-Sizing-Up-Online-Audience-
Measurement-Services.aspx.
Why hasn’t a measurement currency emerged online? The preceding
review of traditional media currencies begins to point to an answer —
first, in the unprecedented abundance of data about online audiences and
behaviors available from multiple sources, and second, in the limited role
third‐party ratings play in planning and paying for online ad campaigns.
The following two sections will review each of these threads in turn,
pointing to a basic decoupling of audience measurement and advertising.
Ultimately this analysis suggests that though a single standard for
estimating audience size may emerge, it won’t play the pivotal role that
measurement currencies have in the past.
II. Measuring Online Media: Disputes About Data
The term ``banner ad'' was coined by the site Hotwired, which
standardized the novel advertising format and began selling it on a wide
basis in late 1994. (By most accounts AT&T was first to try the format,
with a come‐on that read, ``Have you ever clicked here? You will.'')
Almost immediately Hotwired took the logical next step and began to
report on ``click‐through rates'' to its advertisers, giving them a new way
to assess the success of their campaigns.
From its earliest days, the nascent online advertising industry was taken
to herald a revolution, offering a precision and depth of information
unmatched by any other advertising platform. The Internet was touted as
the first truly ``accountable'' medium. A new class of consultants and
agencies sprang into being to develop a vocabulary of techniques to
exploit the medium’s capabilities.
And yet, the reigning perception online is one of chaos and confusion. The
industry cannot agree even on basic conceptual definitions, such as what
constitutes a ``unique visitor.'' In the world of online news, individual
publishers routinely negotiate a number of basic audience metrics which
are not only mutually incompatible, but also vary wildly from month to
month. Publishers seem to agree that much of the available data are
unreliable, but disagree about precisely which.
Reconciling these two basic features of the online measurement world —
abundant information and persistent confusion — is the key to
understanding this world and how it is likely to develop. The chaos of
audience information online does not represent the failure of the Internet’s
promise as an advertising medium, but rather its realization. Information
abundance is chaotic.
An embarrassment of data
In spite of the ceaseless business hype surrounding the Internet, it can be
easy to understate the shift it marked in terms of the quantity and variety
of data generated about audiences. As noted previously, traditional media
operate with a relative paucity of ``natural'' information. Without
conducting surveys, a TV or radio broadcaster has no direct indication of
the number of people watching or listening.
Print publishers have more natural information at their disposal.
Newsstand‐driven publications record the number of copies of each issue
distributed and returned, but have to estimate total readership.
Subscription‐based periodicals also have basic demographic information
about their readers. (Controlled‐circulation titles know much more, and
use reader surveys and free subscriptions to actively shape an audience
profile desirable to their advertisers.)
Source: authors’ research
On the Internet every action by a reader generates a data trail in a chain of
computers running, at a minimum, from the Web site being visited, to the
ISP, to the user’s own browser. Each of these three tiers has become the
basis for a competing approach to audience measurement: Server‐based
(or census‐based) analytics software runs on publisher servers; ISP‐based
estimates collect traffic data from major ISPs such as Verizon and Time
Warner; and panel‐based metrics use tracking software installed on the
computers of a panel of Internet users.
Ad servers and advertising networks introduce additional layers of data
collection: a single page request by a reader may result in calls to the
publisher’s in‐house ad server, to individual third‐party ad servers, and to
an advertising network. Server activity at each of these layers can be
aggregated and analyzed. More importantly, both content and advertising
servers at each layer may introduce a ``cookie'' to identify the reader’s
computer in the future as he or she returns to the current site, or visits
other sites in the same editorial or advertising network. As a result, a
single user’s actions may be simultaneously tracked by multiple categories
of observers.
Over time, then, the number of parties who can produce meaningful
information about online audiences and audience behavior has increased.
The vocabulary of audience‐related statistics has increased with it: To note
only a few of the most basic metrics, publishers and advertisers today
must be conversant in ``page views,'' ``click‐throughs,'' ``unique visitors,''
``usage intensity,'' ``engagement time,'' and ``interaction rates,'' in addition
to the demographic and behavioral profiles of their audience.
The irony of expectations
This flood of data from different sources has resulted in a level of
complexity that can be difficult to manage. The key selling point of digital
media remains the ability to track consumer behavior: which pages were
viewed, which banners were clicked, and when this viewing and clicking
produced an ``action'' such as a request for information or an online
purchase. In practice, though, making sense of the massive amounts of
data collected is hard work.
Tom Heslin, senior vice president and executive editor of the Providence
Journal, calls this the ``irony of expectations'': neither publishers nor
advertisers have been able to keep up with the flood of data. ``Our biggest
challenge is to simplify solutions for our clients, even for national
advertisers,'' he explains. ``The development of metrics has far outstripped
knowledge of ad buyers and sellers. There is a real disconnect between the
technology and how it can be applied and used.''
The main effect of the rising tide of information has been to increase
uncertainty among advertisers as well as publishers, according to Rick
Hirsch, multimedia editor of the Miami Herald. ``Ironically itʹs still like
being a traditional editor making calls based on your gut instinct — you
have more data, but itʹs conflicting,'' he explains.
Hirsch says server data (analyzed via Omniture) gives him some idea of
what share of his overall audience comes from Miami rather than from the
Caribbean or Latino communities in other parts of the U.S. (This is based
on visitors’ IP addresses.) However, he has no way to match that data to
particular stories. As a result, for instance, Hirsch can’t confirm his
suspicion that a core Herald audience consists of government employees
working in Miami, which would argue for augmenting that beat.
Likewise, he is unsure how much to invest in edgy video projects because
he doesn’t know whom they appeal to.
The Web‐native news site Talking Points Memo offered a dramatic
illustration of the abundance of information available today. TPM has
been beta‐testing a new server analytics package, Chartbeat, which offers
a detailed real‐time picture of the last 15 seconds of activity at the popular
site. The software provides an instrument panel with a minute‐by‐minute
picture of what articles people are reading, how far into each piece they
read, which pieces they’re commenting on, what readers are searching for
on the site, who’s linking to the site from elsewhere on the Web, and what
people are saying about TPM on Twitter, among other data.
``I’ve been working on the Web for 15 years, but this blew my mind,''
declares Kourosh Karimkhany, COO of TPM Media. ``It was a real
epiphany.'' Karimkhany says that the real‐time information is having a
dramatic impact on editorial and design decisions, for instance by
revealing exactly where readers drop off in each story (halfway down a
page, there’s almost no audience left) and by challenging expectations
about which breaking stories deserve top billing. For instance, editors
were surprised to see news of Al and Tipper Gore’s divorce (even before a
Portland masseuse came into the story) outperforming the political
bombshell about General Stanley McChrystal’s profile in Rolling Stone,
and moved the divorce story into a more prominent spot.
Measurement companies themselves appreciate the systemic effect
produced by the many kinds of audience data now available. Marketing
copy from comScore concedes the point frankly:
``The frequent disparity between census‐based site analytics data
and panel‐based audience measurement data has long been the
Achille’s heel of digital media measurement. Because the two
measurement techniques have different objectives, they employ
different counting technologies, which often results in differing
metrics that can cause confusion and uncertainty among publishers
and advertisers.''^{\href{#endnotes}{9}}
Marc Johnson, CMO of Experian Hitwise, agrees that more information
hasn’t always yielded greater clarity. ``Digital media is much more
complicated. There are many more things that are measurable. There are
many more moving parts,'' he says. ``It’s not always agreed upon what is
the most important thing to measure, and what those measurements mean
or how they should be applied.''
This complexity does not appear likely to abate in the near future. If
anything, the variety of audience measures available seems to be
increasing as sites and advertisers try to accommodate mobile devices
such as smartphones and e‐books.
Publishers finesse multiple metrics
The result of this abundance has been that unlike their counterparts in
traditional media, publishers working online routinely subscribe to both
panel‐based and census‐based measurement services — that is, to
multiple, incompatible estimates of audience size. Each of the newspapers
interviewed for this report (though notably not the online‐only news site,
TPM) subscribed to either Nielsen or comScore, or to both, while also
relying on a server‐side Web analytics package, usually Omniture. Most
also incorporate at least one additional source of audience data, such as
Scarborough, Hitwise, Google Analytics, or Alexa, into their internal
analysis and their pitches to advertisers.
However, impressions of the relative merits of these data sources vary
widely. Some publishers find server‐side data much more reliable. The
Providence Journal subscribes to comScore, but sees hard‐to‐credit
fluctuations in its online audience from quarter to quarter. As a result, the
paper relies on audits of its server traffic, collected via Omniture, to come
9 http://comscore.com/About_comScore/Methodology/Unified_Digital_Measurement.
up with its official online readership. The Journal relies on comScore data
mainly for ``product development'' — to gauge the success of niche sites
among particular demographic targets.
Similarly, the Miami Herald uses comScore and also used to subscribe to
Nielsen. But Hirsch reports that his paper’s position in either of the panelbased
rankings varies for no apparent reason. ``I don’t know when to
believe them,'' he says. Meanwhile traffic recorded by the Herald’s own
servers, analyzed with Omniture, tends to match his own editorial sense
of when certain stories, or entire editions, are commanding a great deal of
attention in Miami.
As an example, Hirsch points to January of 2010, the month of the Haitian
earthquake that claimed an estimated 230,000 lives. ``We know our traffic
went through the roof, because of our history of coverage in the region,''
Hirsch says. The paper’s internal figures matched expectations: as
recorded by Omniture, traffic spiked 36 percent over December, to 35
million pageviews, while unique visitors jumped 11 percent, to almost 6
million people. Meanwhile, though, comScore recorded less than half as
much traffic for January, and fewer than an third as many unique visitors.
percent the month of the earthquake — and falling again in February,
despite the fact that Miami hosted the Super Bowl that month.
At larger national papers, the story is somewhat different. The Washington
Post relies on both comScore and Nielsen data to understand how it fares
against major competitors online, while also using server‐side traffic
figures for internal strategic analysis. Managing editor Raju Narisetti
acknowledges that the cacophony of competing measurements has been a
serious issue, with both the panels undercounting the Post’s audience.
``However, over time you can recognize the strengths and weaknesses of
each and start to understand how one approach relates to the other,'' he
explains. ``It is less of an issue now'' — in part because of the hybrid
measures in the works from comScore and Nielsen, which bring these
audience estimates closer to internal data.
The Wall Street Journal subscribes to Nielsen, comScore, and Hitwise, in
addition to using Omniture for server‐side analysis. Kate Downey, the
paper’s director of ``audience analytics & insights,'' observes that the
Nielsen and comScore ratings of wsj.com rarely agree with each other, or
with the Journal’s own records. However, she emphasizes that server data
is also unreliable and prone to double‐counting; to make their case to
advertisers, salespeople rely mainly on demographic data from the panels
and on the Journal’s own registration records (all the more valuable since
much of the site is behind a paywall). She appreciates having multiple
sources of data at her disposal, each with its own strengths and
weaknesses, and suspects that many of her peers at other papers agree.
``People use whatever numbers look good that month. It gives publishers
some flexibility,'' Downey explains. ``I think if everybody had the same
numbers, we would hate that even more.''
Talking Points Memo sees the same dramatic divergence in audience
estimates. Google Analytics counted 1.8 million unique visitors for a
recent 30‐day span, while comScore typically gives it in the neighborhood
of 300,000 visitors per month. But unlike its peers in the newspaper
business, TPM’s response is to ignore the panels outright — the site
subscribes to neither comScore nor Nielsen, counting on advertisers and
agencies to supply panel‐based figures if they consider them necessary to
the conversation. (For demographic data, TPM relies on its own, voluntary
audience surveys; every six months or so founder Josh Marshall issues an
appeal to readers, culling about 1,000 responses within 12 hours.)
``The panel‐based numbers are atrocious,'' says Karimkhany flatly,
pointing out that most of TPM’s traffic comes from the workplace, which
the panels don’t capture well. ``But as long as they’re equally inaccurate
for our competitors, it’s okay. It’s something we live with.''
The controversy over ``unique visitors''
Within the measurement industry, this overabundance of information
works itself out in periodic disputes over data — disputes over what
information is most important, and over best how to define or collect it.
The fault line that surfaces most frequently is between panel‐based and
server‐side measures. The current agitation for new standards (detailed
below) springs in part from a very public disagreement in 2009 over what
might be fairly called the atomic particle of online audience measurement,
the ``unique visitor.''
For the first decade of online advertising, the total number of unique
visitors to a site was usually defined as the count of unique ``cookies,'' deduplicated
over the period of analysis. This had become the de facto
standard since most sites don’t require a log‐in or authentication. But, it is
a ``technology‐based'' rather than a ``people‐based'' standard. A single
user visiting from multiple computers (or deleting cookies from his or her
browser) will inflate the count; multiple users sharing a computer will
produce undercounts.
In 2006 the Web Analytics Association, representing mainly server‐side
measurement firms, published a definition of unique visitors that added
the option to use ``authenticated users'' when available. The precise
meaning of this new standard was unclear; according to a recent article in
Mediapost, ``the goal of the standard was to educate the Web analyst to the
most commonly used definition and to encourage vendors to openly
document any variances from the standard, given that data collection and
processing techniques may vary from vendor to vendor.''^{\href{#endnotes}{10}}
In 2009 another trade group, the Interactive Advertising Bureau,
published a competing definition of ``unique users'' aimed mainly at
panel‐based measurement services such as Nielsen and comScore, but
specifying that census‐based tools such as Google Analytics should
conform as well. The new guidelines require that the measurer ``utilize in
its identification and attribution processes underlying data that is, at least
in a reasonable proportion, attributed directly to a person.''^{\href{#endnotes}{11}}
The definition touched off a heated debate and drew heavy criticism for
being overly vague. The IAB’s standard invited a new set of questions:
Will sites be required to collect personally identifiable information? What
would the privacy implications be for sites adopting this definition? Is this
new guideline ultimately even applicable to Web analytics firms, or can it
only be met by audience measurement companies?
As a result, what had been a fairly straightforward metric — if one whose
relevance was sometimes questioned — now has multiple definitions,
10 J. McDermott, ``Unique Visitors Come In Two Shapes And Sizes,'' Metrics Insider,
20 March 2009, at http://www.mediapost.com/publications/?fa=Articles.showArticle
&art_aid=102512.
11 Interactive Advertising Bureau, ``Audience Reach Measurement Guidelines,'' 23
February 2009.
used in multiple ways, by multiple firms. The episode suggests that online
measurement is hamstrung not only by the abundance of data available,
but also by the inevitable contentiousness of even well‐intentioned efforts
to define standards in a developed industry. The IAB and WAA have been
working together to approve (though not to adopt) each other’s definition
of unique visitors, but they have yet to reach a consensus.
Well‐known methodological weaknesses
Despite such disputes, the strength and weaknesses of various approaches
to audience measurement are well known and widely agreed upon. This is
especially true in the comparison between panel‐ and census‐based
metrics.
Assuming the panel is well‐built, panel‐based measurement offers two
key advantages over server‐side data. First, a panel permits demographic
analysis, allowing a national news outlet to determine, for instance, its
penetration among men in a certain age or income group. Second, panelbased
research facilitates comparisons across competing sites and over
time — knowing whether an outlet is improving its position vis‐à‐vis the
competition.
Nielsen provided basic details about its methodology for this report. The
company’s Internet panel consists of about 200,000 people recruited online
through various partners; panelists agree to install metering software that
tracks their online activity on a click‐by‐click basis. To correct for biases
among panelists recruited online, Nielsen uses a ``calibration sample''
culled via traditional offline techniques such as ``random digit dialing''
(some of these users are drawn from the 18,000 households in its TV
panel). Nielsen is able to report on panelists Internet usage on a monthly,
weekly and daily basis.
The validity of panel‐based measures depends on how faithfully they
reflect the larger public. Tracking software installed on member’s
computers has difficulty distinguishing between multiple users, and far
more important, it misses what its members do on other computers —
especially at school or in the workplace. As a result, sites that target
professionals during business hours, such as the Wall Street Journal and
Talking Points Memo, believe they are underreported since workplaces are
reluctant to participate in the panels.
This basic flaw is widely recognized. ``The over‐reliance on panels whose
members accept tracking software on computers has been seen as an
acceptable way of measuring audience size,'' explains the Washington
Post’s Raju Narisetti. ``However, this approach is non‐random and violates
the standards required to project to larger audiences with any degree of
certainty. Further, there has been no effort made to determine alternative
means of measuring usage for people who do not accept the software
(such as government employees, companies with privacy policies, etc.).''
Meanwhile, having two major firms offering panel‐based ratings exposes
methodological inconsistencies and makes it much more difficult to use
the ratings as a benchmark. Nielsen and comScore frequently disagree
about even basic measurements — such as who is the No. 2 media
company online, after Google. Per comScore, Yahoo is the runner‐up, with
167 million unique visitors in May 2010. But that month Nielsen had
million visitors to Yahoo properties — a difference of 34 million people,
about the population of Canada.
Source: Interactive Advertising Bureau analysis
The Interactive Advertising Bureau has drawn attention to differences in
the site rankings released by Nielsen and comScore, for instance in a slide
on traffic in the ``news and information'' category, reproduced above.
Some confusion results from way sites are grouped: comScore rolls up
properties like Nytimes.com and About.com into ``New York Times
Digital,'' while Nielsen counts them separately. But even apples‐to‐apples
comparisons can differ widely. In May 2010, all Gannett‐owned properties
commanded 37.5 million unique visitors according to ComScore, but just
25.6 million according to Nielsen. The same month ComScore gave
washingtonpost.com an audience of 17 million people, but Nielsen
recorded fewer than 10 million.
The advantages of server‐side measurement are similarly straightforward:
Web analytics can claim to capture everything that happens at a given
publisher’s site, in census‐like fashion, with no need for dubious
extrapolations. In addition, server‐based measurement offers a level of
behavioral detail panels cannot hope to match, following how individual
readers make their way through an online publication, how much time
they spend on each article, and so on.
TPM’s Karimkhany argues that statistical panels are an anachronism
when concrete and comprehensive traffic data is available from Google
Analytics and its ilk. ``I have real concerns about Google’s market power,''
he says. ``But there is no reason not to trust Google Analytics. It’s a rocksolid
product, everybody uses it, and it’s very difficult to game.''
The core technical flaw in server‐side measurement has been noted earlier:
It tracks machines (or actually ``cookies'') rather than people, and so is
highly vulnerable to miscounts when human either share computers, or
use multiple computers and browsers. People who delete cookies may be
counted more than once. Another major challenge is eliminating nonhuman
visits to a site, especially from ``bots'' or ``spiders'' which search
engines use to crawl the Web.
As important as these technical issues is the fact that advertisers and ad
agencies tend to disregard server‐side data. A study prepared by Bain &
Co. for the Interactive Advertising Bureau in 2009 emphasized brand
advertisers’ dissatisfaction with the server‐side metrics publishers
generally make available, such as ``page views,'' ``time spent on page,''
and even ``unique visitors.''^{\href{#endnotes}{12}}
For this reason, the NAA’s Randy Bennett suggests that publishers have
made too much of the discrepancy between server‐side and panel‐based
measurement. ``They tend to focus on the discrepancy between metrics
rather than on what advertisers want,'' says Bennett. ``In the end it doesn’t
matter what publishers want. It only matters what advertisers want. And
there’s no standard around that.''
Finally, advertisers as well as media providers routinely bemoan the
inability to measure audiences in a consistent way across multiple
platforms — television, the internet, mobile phones, etc. In 2005, a report
by the Advertising Research Foundation found that ``many respondents
included comments about the lack of multi‐media comparability and
difficulties that they experience in integrating data from the measurement
12 Bain & Co., Building Brands Online: An Interactive Advertising Action Plan,
November 2009.
of various media for which they provide integrated planning support.''^{\href{#endnotes}{13}}
And in 2009, a new ``Coalition for Innovative Media Measurement'' — led
by giants such as CBS, NBC, and Disney — was formed to establish a new
standard to gauge total media usage across broadcast and the Internet.
The promise of hybrids
The online media measurement landscape evolves quickly. While
comScore and Nielsen remain the two top panel‐based services, others are
trying to encroach. Methodological differences have eroded between the
two panels and competitors who derive their data partly from ISPs, such
as Hitwise, Compete and Quantcast. According to Hitwise’s Johnson,
``Everyone is trying to grow their business by offering a full suite of data
to marketers. More and more we overlap in each other’s areas.''
Meanwhile, both Nielsen and comScore are adopting a ``hybrid model''
which combines their panel research with server‐side data collected from
clients. comScore’s Media Metrix 360 is the first such offering, a ``panelcentric
hybrid'' that combines the company’s two million person global
panel with server‐side analysis. The goal is to deliver a unified count that
reconciles discrepancies between panel and server data, as well as to
provide more granular detail on Web‐site usage.
Nielsen’s version has not been officially unveiled, but interviews with the
firms and their clients indicate that both hybrids work similarly: Client
sites embed ``beacons'' on their content servers that allow Nielsen or
comScore to track visits from users who aren’t members of their panels.
(comScore’s beacon has been integrated directly with Omniture’s popular
Web analytics software.)
How are these conflicting data sources reconciled? Per comScore’s site, the
firm ``has developed a proprietary methodology to combine panel and
server‐side metrics in order to calculate audience reach in a manner that is
not affected by variables such as cookie deletion and cookie
blocking/rejection.'' Or as a Nielsen analyst explained, ``The essence of
what we’re doing is creating ‘person‐centric’ audience measurement data
using the strengths of panel‐based measurement (quality demographics)
and server‐side data (census‐level tracking of content).''
13 Cited in The Future of TV.
Whatever its technical merits, the immediate effect of the hybrid approach
has been to increase audience figures for many sites, pushing panel‐based
figures closer to publishers’ own internal estimates. According to one
comparison^{\href{#endnotes}{14}} based on comScore’s December 2009 data, unique visitor
counts went up an average of 30 percent under the hybrid approach; some
sites — The Onion is one — saw traffic nearly triple. (Not every site has
been so lucky, however; according to a recent New York Times article, a
methodological tweak at comScore slashed Hulu’s traffic by 45 percent in
June.^{\href{#endnotes}{15}})
The boost has been especially dramatic for newspapers, according to
comScore’s Josh Chasin. (A discrepancy of 75 percent had not been
uncommon for news outlets, due most likely to their high at‐work traffic.)
Both the New York Times and the Providence Journal report that the hybrid
figures better reflect their own audience estimates. The increase has been
substantial: comScore’s audience estimate for Times properties jumped to
72 million in May 2010, up from 53 million in December 2009, before the
new methodology was implemented (and up from 47 million in May
2009).
While hybrid measurement promises more reliable audience estimates,
though, it is not clear that it will result in a single audience standard
online. The new methodology adds another layer of complexity, and its
implementation has been piecemeal: Sites that do not download and
install beacon software on their servers cannot be measured with the new
hybrid formula, and therefore should not be compared directly to sites
that do participate — even though the measurement firms purport to rank
all sites in the markets they cover.
This landscape is further complicated by the higher level of access
afforded to paying clients. After initially limiting the service to its
customers, comScore now allows any site to install a beacon, for free; it is
not clear whether Nielsen will follow suit. According to Jon Gibs, vice
president of analytics at Nielsen, ``We can’t do things just for the good of
the industry. If there’s no one paying for the service, it doesn’t make sense
to do it.''
14 Analysis by campaign management firm XA.net; see http://www.cpmadvisors.com
/2010/02/06/comscore-hybrid-methodology-leads-to-large-traffic-increases/
15 A.R. Sorkin & M.J. de la Merced, ``Hulu Is Said to Be Ready for an I.P.O.,'' New
York Times, 8 August 2010.
Reform in the air
The controversy over defining unique visitors — and in general over the
very different pictures of the online world painted by various
measurement firms — has provoked calls for an organized, industry‐wide
reform. The state of affairs was captured well in a column by Steve Smith,
digital media editor at the Media Industry Newsletter:
``We are a decade and a half into the life of the ‘most accountable
medium,’ the Web, and just this week we see some of the major
online measurement firms still tweaking their models and arguing
over methodology. I have major media companies reporting their
monthly numbers to me, and I see staggering differences between
the stats they claim from their internal logs via Google Analytics or
Omniture and the third‐party numbers. Itʹs not just mobile, either.
Itʹs still a mess all over''^{\href{#endnotes}{16}}
In a sign of the times, the Internet Advertising Bureau — a trade group
``dedicated to the growth of the interactive advertising marketplace'' —
made Smith’s condemnation the opening slide of a presentation in early
February to the Association of National Advertisers. (The presentation
went on to highlight the wide differences in the monthly rankings
released by Nielsen and comScore, reproduced above.)
IAB appears to be leading the reform charge, at least in the world of
panel‐based measurement. A confidential McKinsey & Co. study
commissioned by the IAB and the American Association of Advertising
Agencies concluded in 2009 that confusion over metrics stood in the way
of greater online ad spending. Based on this report, the IAB has proposed
a ``cross‐industry task force of senior marketers, agency executives and
media leaders to reach consensus on standardizing and simplifying basic
audience measurement.'' The Newspaper Association of America has
committed to support this task force and to recruit participants from the
newspaper industry.
A parallel standardization effort is underway from the Media Ratings
Council, a group comprising media companies, advertisers, and agencies
which dates from the 1960s and whose mission is to accredit and audit
16 S. Smith, ``Taking Mobile's Measure, Again... And Again,'' Mobile Insider, 26
January 2010.
ratings firms. The MRC has worked with the IAB to coordinate definitions
(for instance of IAB’s controversial 2009 guidelines for ``unique visitors'').
However the MRC is also in the process of accrediting comScore and
Nielsen, an effort likely to continue into next year. It is conceivable that
these ongoing audits will make the two panel‐based measures more
compatible, and perhaps more transparent to publishers and advertisers
using them.
In the world of server‐side measurement, any number of organizations
offer third‐party auditing of traffic data from Omniture and other Web
analytics tools. Both leading circulation auditors, ABC and BPA, operate
interactive arms that produce verified online readership figures. ABC has
been particularly active here: Its ``Audience‐FAX'' service, developed in
partnership with the NAA and Scarborough, purports to measure a
newspaper’s net readership across print and the Internet. However, this
service depends on newspapers to submit readership data in various
categories; the competitive picture may be skewed if newspapers calculate
these differently.
The emergence of new platforms and devices also poses a growing
problem. For instance, as yet no consensus exists around how to measure
streaming media and online video. How is an impression defined when
content is continuously streamed? When it is short‐form versus long
form? When it plays in the background, as most online radio does today?
How these terms are defined is no small matter – streaming music service
Pandora today stops playback if a listener doesn’t engage with their page
in a certain amount of time. Each time a user has to click back to that page
to hit ``play,'' a new session is initiated, thereby boosting Pandora’s traffic
numbers.
Will an audience measurement currency take hold online?
A number of countervailing forces appear to be at work in the online
media measurement today: on one hand, explicit standard‐setting efforts
and the emergence of ``hybrid'' audience measures, and on the other, a
growing diversity of measurement companies, media types, and
technology platforms.
Two points bear consideration. The first is that the most successful media
measurement currencies have emerged not from industry task forces, but
through market power, which is to say research monopolies. The single
development which would do the most to clarify audience measurement
standards would be a union of comScore and Nielsen NetRatings — an
event which, given the high costs of maintaining competing panels and
the obvious benefit of eliminating embarrassing discrepancies, is not out
of the question.
Combined with the assimilation of server‐side data via ``hybrid''
approaches, such a union would establish a single, industry‐wide
standard for comparing online audiences. Alternatively, a server‐side
measurement may take hold as an audience standard if (as TPM’s
Karimkhany suggests) the industry is gradually coming to see panel
measures as obsolete. In this scenario, the most likely candidate for a
standard is Google Analytics, which is much cheaper and far more
widespread than alternatives like Omniture, especially among blogs and
smaller, independent Web sites. (Google also has the technical advantage
of being able to calibrate online measurements using its own vast
audience and huge advertising network.)
The second point to consider is that even if a consensus emerges by either
of these routes, the resulting standard will not automatically be a media
``currency'' in the way of Arbitron’s radio rankings or Nielsen’s TV
ratings. Agreeing on a single estimate (however imperfect) of the number
of people who read the Times online last month, or on whether the Times
or the Post did better among high‐income women, will not stanch the flow
of information about what Web users are doing and what they care about.
It will not prevent either of those papers from touting other statistics
which strengthen their case. And most important, it will not necessarily
satisfy the needs of advertisers, who no longer have to plan their
campaigns — nor pay for them — on the basis of static readership
profiles. The following section will investigate how media metrics are
used in planning and executing advertising campaigns, and how this
affects publishers.
III. Measuring Online Media: A New Planning Paradigm
Third‐party research plays a critical role in the offline media ecosystem.
As a measurement currency, information from monopoly ratings firms
such as Nielsen and Arbitron guides media planning, governs media
pricing, and is even used to assess the success of ad campaigns. However,
audience measurement plays a substantially diminished role in
advertising on the Internet. Two shifts help to account for the evolving
role of media measurement online, and are explored below: the
emergence of performance‐based pricing models, and the increasing
reliance on new kinds of audience targeting.
An online Gestalt switch
The role of media measurement in online advertising reflects the basic
shift in the way media space is bought and sold on the Internet. One way
to appreciate this Gestalt switch is to consider what ``inventory'' means
online and off. In broadcast or print advertising, inventory is a scarce
resource — no matter its circulation, a newspaper has a finite number of
ad pages in each edition at a reasonable ad‐edit ratio. This scarcity is even
more pronounced in broadcast; hence the practice of reach ``guarantees''
promised to advertisers on the basis of past Nielsen ratings, and adjusted
after the fact (via ``makegoods'' or ``overdelivery'') once Nielsen results are
in for a given campaign.
Online inventory cannot be a scarce resource in the same way, since it is
generated on the fly by each decision to view a page. In theory, it should
be unnecessary to speak of audience guarantees at all on the Internet — an
ad banner or pop‐up can simply be shown until the purchased number of
impressions has been reached. (In practice the most desirable online
property is often sold as sponsorships, not impressions; and even
impression‐based campaigns will prefer outlets with a large enough
audience to deliver the desired audience within a certain time frame.)
This shift is well‐understood, of course, but what it means for media
measurement has not always been appreciated. Advertisers purchasing
space or time in traditional media are paying, in a very immediate way,
for a set of audience numbers delivered by a trusted third party. Based on
Nielsen or Arbitron figures, applying the standard formula of ``reach and
frequency,'' a company like Proctor & Gamble can calculate (if with
disputed accuracy) what it will cost, say, to make sure 40 percent of TV
viewers or radio listeners in a certain market hear a Duracell jingle an
average of three times each.
Online such calculations are both less possible and, to many advertisers,
less relevant. Neither user panels nor server‐side analytics can realistically
claim to gauge the ``reach and frequency'' of an ad campaign across
multiple sites and ad networks. Ad impressions have been decoupled
from media audiences. People reading the same article online will not
necessarily see the same ad banner, making the link between a media
property’s reach and that of its advertisers much more tenuous. More to
the point, advertisers no longer need ``reach and frequency'' to plan
campaigns or purchase media; they no longer need a measurement
currency in the same way.
Marc Frons at the New York Times makes this point succinctly, discussing
the disagreement between different rankings of top online news outlets. ``I
think itʹs less important online because advertisers can see how well the
ad is performing on their end,'' Frons says. ``So the Nielsen number and
the comScore number are just bragging rights for publishers. They matter
less.''
Pricing online ads: impressions versus performance
Online as in traditional media, the basic unit for pricing advertisements
remains the CPM, or cost per thousand. Its persistence has defied
predictions that performance‐based pricing schemes would sweep aside
the old‐media habit of selling exposure.
However, the online CPM differs from its offline cousin. In broadcast,
CPM is based on households or viewers, and in print on audited
circulation; thus an advertiser can roughly compare the cost of reaching
1,000 people via a TV spot and a magazine spread. Online, though, CPM
refers to impressions rather than viewers or readers, making cross‐media
comparisons difficult. One session at a news outlet online may generate a
dozen impressions as the reader clicks around from story to story.
The most sought‐after inventory online is usually sold on a CPM basis.
However, several other pricing models exist, including CPC, or cost
per click; CPL, or cost per lead (usually determined by a user registering
for a newsletter, account, etc.); or CPA, meaning cost per action or cost
per acquisition (for users who convert to customers). Less desirable
inventory is often sold on the basis of performance, so the advertiser only
pays for the desired result.
These distinctions may not be as defining as they are made out to be. An
advertiser who buys media on a performance basis knows the total
number of impressions delivered and can easily calculate CPM, for
instance in order to compare two sites on an apples‐to‐apples basis. The
reverse is true as well — an advertiser who buys on a CPM basis also has
records of click‐throughs and purchases and so can derive the various
performance measures. And again, all of these approaches differ
fundamentally from offline cost‐per‐thousand deals in that on the
Internet, the thousand impressions (or clicks or actions) are recorded one
by one, not based on audience estimates.
Two tiers of online inventory
From a publisher’s perspective, online pricing schemes creates a clear
caste system based on the distinction, inherited from offline media,
between premium and remnant inventory. Online the two categories are
not formally defined, but still fairly clear:
\begin{itemize}
\item Premium inventory sells for a relatively high rate; it is usually sold
\end{itemize}
by the media owner, rather than a third party like an ad network; it
is more often the province of ``brand advertisers''; and it typically
sells on a CPM basis or as a sponsorship package.
\begin{itemize}
\item Remnant inventory sells at a low rate; deals are usually transacted
\end{itemize}
through an ad network or aggregator, often on a performance basis
(CPC or CPA); and buyers, who hail from the direct‐response end
of the spectrum, may have little idea where their ads end up
running.
Talking Points Memo offers a good illustration. The site sells roughly onethird
of its inventory direct, one‐third as remnant, and the final third
through Google’s AdSense, though the ratios fluctuate from month to
month. TPM’s Karimkhany suggests that premium inventory might
command a CPM of roughly \$10. Unsold or less desirable inventory is
offloaded through remnant optimizers for a much lower price, ranging
from perhaps 40 cents to \$2. Meanwhile inventory sold through Google’s
AdSense network varies even more unpredictably depending on
advertiser demand for a given keyword, from an effective CPM of about
\$2 to as much as \$20 in extreme cases.
``All this is very dynamic,'' Karimkhany explains. ``Sometimes we sell a lot
of direct ads, such as when BP needed to get its message out and
environmental groups wanted to give the counter‐message. This crowds
out AdSense and remnant. Sometimes AdSense goes crazy, like during
elections when campaigns buy keywords. And sometimes, like January
and February and summer vacation months, remnant predominates
because thereʹs very little active purchasing.''
It is entirely possible for an advertiser to buy inventory on a single site
that is both premium and remnant, if the advertiser is negotiating directly
with the site and also working through an ad network. However, top‐tier
sites often have a policy against selling any inventory as remnant.
This is the case at both the New York Times and the Wall Street Journal,
which sell most online inventory on a CPM or sponsorship basis and do
not participate in ad networks (other than Google’s AdSense, which the
Times uses). ``We sell brand, not click‐through,'' declares the Journal’s Kate
Downey flatly. ``We’re selling our audience, not page counts.''
Marc Frons echoes the sentiment, pointing out that the Times can afford to
take the high road. ``For us as the New York Times, brand is important,'' he
says. ``You really want to make the Internet a brand medium. To the
extent CPC wins, thatʹs a bad thing.''
Other newspapers take a hybrid approach. The Providence Journal
negotiates cross‐platform, multimedia packages whenever possible — for
instance, combining a quarter‐page print ad with a certain number of
banner impressions, a search term, and (via a partnership with Yahoo!) a
behavioral targeting profile. These bundles arguably help to resist any
erosion of ``projo.com'' into a second‐class, performance‐based ghetto.
However because traffic is hard to predict and can vary greatly from
month to month, the paper unloads ``oversupply'' as remnant inventory
through Yahoo’s ad network.
A new dynamic in media planning
A basic reason syndicated research plays such a critical role in broadcast
and print media is that most advertisers have no good way to judge the
effectiveness of their campaigns. Media planners thus frontload most of
their analytical time and resources, using demographic data — imperfect
as it is — to plot and plan campaigns before they run. To a great extent in
traditional brand advertising, once a campaign launched the media
planner’s work was done, until the post‐mortem.
That media‐planning dynamic is inverted online. Advertising agencies as
well as in‐house marketers tend to allocate more resources to optimizing
campaigns as they run than to planning them beforehand. Much has been
made of the cultural and philosophical shift this required at major
agencies, which have had to import the thinking of the much less
fashionable world of direct marketing.
Source: IAB and Bain & Co., Building Brands Online: An Interactive
It is still true that direct‐response advertisers are more likely to use
performance‐based pricing online, while brand advertisers gravitate
towards CPM. But there are no hard and fast rules, and even brand
advertisers have an irresistible stream of data about how their campaigns
are doing. Taken to a theoretical extreme, an advertiser concerned only
with results would have no need for up‐front demographic data at all, or
indeed for media planning as it has traditionally been understood. When
the only metric that matters is sales, who sees the ad or how often
becomes irrelevant.
Ratings define the ``consideration set''
Because of this inverted planning dynamic for online advertising, the role
of syndicated research services like comScore and Nielsen is greatly
diminished. Rather than defining a currency, these panel measurement
companies act as a sieve, helping planners to come up with a list of sites
and ad networks to potentially include in a campaign — the
``consideration set.''
As a result, a site like the Wall Street Journal pays close attention to its
ranking and especially to its demographic profile in the syndicated
research services. ``We live and die by the demographics,'' says Downey.
Once a potential advertiser has the Journal on its list, however, salespeople
can make their case using other research sources, including the
newspaper’s own records. This often means ``educating'' potential
advertisers about the under‐representation of the Journal’s valuable atwork
audience on the major research panels.
While a good comScore or Nielsen profile may make a site attractive to a
brand advertiser, the audience data are not used to set the terms of an ad
buy. How a deal is structured will depend more than anything (and more
than in offline media) on the specific objectives of the advertiser. But even
in a straightforward CPM deal, media performance matters. If a site does
not deliver desired impressions (or clicks, or conversions) quickly enough,
it runs the risk of being cut from the media schedule.
In fact, cutting venues out has become routine campaign strategy. ``You
want to cast a fairly wide net initially, then refine it as the campaign
runs,'' explains Rudy Grahn, director of analytics at media agency Zenith
Optimedia. Advertisers have other tactics at their disposal, including
renegotiating the price or the volume of impressions, but a site that
consistently under‐delivers will find itself eliminated from a campaign.
The threat from targeting
If one promised ``revolution'' in advertising online has been the ability to
pay only for ads that perform, the other has been the ability to aim
marketing messages at certain individuals with a precision not possible in
the offline world. In fact three dominant models of targeting exist on the
Internet, two of which have direct analogues in traditional media:
\begin{itemize}
\item Demographic targeting uses characteristics such as age, gender,
\end{itemize}
income, and education to define a desirable audience. Most
traditional advertising is targeted by demographic (and
occasionally ``psychographic'') profile at least in the selection of
venues, and the same logic applies in well‐known destinations
online. If BMW advertises on the Wall Street Journal’s site, at
least one reason is its audience of high‐income, educated
professionals.
\begin{itemize}
\item Contextual targeting based on the editorial focus of the venue
\end{itemize}
also routinely figures in media planning both online and off. If
BMW buys impressions on Kelley Blue Book or Edmunds.com,
it hopes to capture potential car buyers already in the decisionmaking
process.
\begin{itemize}
\item Behavioral targeting has no traditional precursor; it relies on
\end{itemize}
cookies to sort Internet users based on what sites they’ve
visited, what search terms they’ve used and what actions
they’ve taken. If BMW wants to target people already in the
purchasing ``funnel'' it can buy profiles of car‐seeking
consumers from a ``profile exchange'' such as BlueKai. Such a
profile might be based on visits to an automotive comparison
site, on a search for ``Mercedes dealers in New York,'' or on preapplying
for a car loan, for instance.
Some high‐traffic publishers are able to deploy sophisticated targeting
technologies in‐house. Dow Jones uses all three approaches described
above to serve ads across its network of business‐and‐finance sites. Two
Wall Street Journal subscribers visiting wsj.com — say, a Cleveland‐based
small business owner and a Manhattan financier — may see different ads
based on anonymous profiles derived from their registration data. (The
profiles are encoded in cookies attached to their browsers.) Or, someone
who frequently visits the ``technology'' pages of wsj.com may be assigned
to a profile that draws tech‐related ads, even as the same user travels to
marketwatch.com or barrons.com.
However, most behavioral targeting occurs through an ad network like
AudienceScience or a profile exchange such as BlueKai. For behavioral
targeting to be predictive, it generally requires wider inputs than a single
site can provide; the goal is to assign users to particular profiles (i.e., ``golf
fanatic'') based on their behavior across a wide swath of the Internet, and
then to be able to reach those users wherever they go. BlueKai calls this
``audience portability,'' and makes no bones about the disintermediating
effect on publishers, or ``inventory'':
``Today, media buying is constrained to only buying data that is
tied to a particular inventory. \ldots  At BlueKai, we understand thatʹs
not the way things should be. We know that online data should be
separated from the media — and made customizable and accessible
to everyone, at scale, anywhere on the Internet.''^{\href{#endnotes}{17}}
As a result, the rise of behavioral targeting represents a distinct threat to
publishers: By discriminating among users individually, behavioral
targeting diminishes the importance of a site’s overall brand and audience
profile. Suddenly the decisive information resides not with the publisher
but in the databases of intermediaries such as advertising networks or
profile brokers. A similar threat may be emerging in the domain of
demographic targeting. As it becomes more possible to attach permanent
demographic profiles to individual users as they travel the Web, the
selection of outlets will matter less in running a campaign.
This is why online media outlets tend not to participate in third‐party ad
networks if they can avoid it. ``We donʹt want to be in a situation where
someone can say, ‘I can get you somebody who reads the Wall Street
Journal while theyʹre on another site that costs half as much,’'' explains
Kate Downey.
17 See http://www.bluekai.com/free-the-data.html.
In fact, Dow Jones is looking at ways to not just resist the trend, but
reverse it, by pulling outside behavioral or demographic profiles into its
own ad servers. This would let the Journal and its sister sites target ads
based on what their visitors — even unregistered ones — have done
elsewhere on the Internet. ``The goal is to be able to give a better ad
experience and user experience to even the anonymous people coming to
our site,'' Downer says. ``That would be fantastic.''
Publishers may also try to control contextual targeting by packaging their
inventory as ``content channels'' designed to respond to advertiser
priorities. For instance, when automotive inventory was at a premium,
many sites rushed to create new sections to draw advertising from car
manufacturers, in some cases pushing beyond their normal editorial
range; a good example is Forbes’ Luxury Car channel online.
Publishers need competitive intelligence, local data
The advent both of performance models and of individual targeting
makes third‐party audience metrics (at least as traditionally understood)
somewhat less decisive for advertisers, and argues against the emergence
of a single online currency on the order of Nielsen’s TV ratings.
However, third‐party metrics remain relevant for both buyers and sellers
of media online — especially for the publishers, who express a clear need
for research to clarify their competitive position, make editorial decisions,
and present a case to advertisers. ``Data on the competitive landscape is
missing, and itʹs important for both editorial development and for ad
sales,'' declares the Miami Herald’s Hirsch.
The two syndicated research services, Nielsen and comScore, appear to be
the best option for this sort of competitive intelligence, despite publishers’
many complaints about the panel‐based measures. This is in part because
publishers are reluctant to advertise their server‐side data, or to trust
claims made by their competition. As the New York Times Marc Frons puts
it, ``Sharing our data with competitors is not something we’re eager to
do.''
What newspaper publishers say they want, almost without exception, is
more demographic granularity. The hybrid measures in development
from Nielsen and comScore hold some promise in this regard. Publishers
who have experimented with the hybrids report that thus far they have a
limited (if useful) objective, to provide more accurate audience counts.
However it seems clear that the measurement firms aim to marry
demographic data more closely to specific newspaper sections. Hirsch
argues that this sort of information would be enormously useful:
``There are a lot of optional, high‐cost, high‐effort editorial projects
a newspaper can choose to pursue. I wish I had the data to guide
these editorial choices. Ironically itʹs still like being a traditional
editor making calls based on your gut instinct — you have more
data, but itʹs conflicting. Better data would make it easier not to
resort to cheap tricks to spike traffic.''
However, especially for regional newspapers such as the Herald and the
Providence Journal, this section‐level demographic data has to be local to be
really useful. The panel‐based measurement tools typically don’t have a
large enough local sample outside of major metropolitan regions to be
able to adequately report on traffic for local newspapers’ web site. It is
partly for this reason that both newspapers and sites like Talking Points
Memo rely heavily on registration data and reader surveys to understand
their audience.
The lack of local insight affects editorial decisions as well as a publisher’s
appeal to local advertisers. ``Our philosophy is we want to sell some
national stuff,'' Hirsch continues. ``But the advertisers who are
traditionally important to us, like department stores, upscale restaurants
— if I can tell them in a compelling way what our audience looks like,
thatʹs good. We need to show upscale professionals are our readers. We
need to distinguish our audience.''
Finally, today’s metrics landscape almost completely neglects the smaller,
independent online news outlets so often credited with representing a
crucial trajectory in the future of journalism. Even a major site like Talking
Points Memo barely shows up in the comScore or Nielsen rankings;
smaller, local journalism projects are off the radar completely.
Based on a series of summits with independent media producers in seven
cities, Tracy Van Slyke of the Chicago‐based Media Consortium asserts
that these independent online journalists are very poorly served by
available data sources. Most cannot afford comScore or Nielsen and are
too small to be captured by panel methodology. As a report produced on
the basis of the ``Impact Summits'' noted,
Many summit attendees expressed frustration with the
inconsistency of current social media analysis schemes.
``Dashboards'' — which combine and analyze a range of data
points on one screen — are in wide use across the online media
environment. For example, web traffic analysis tools such as
Google Analytics or Mint allow webmasters to track the numbers of
site visitors over time, page views, and time spent on a site. \ldots 
However, many public interest media projects are not only too
small to show up on these larger comparative services, but are
increasingly using social media sites like Facebook, Twitter and
YouTube to distribute content, rather than centralizing their work
on a single site. 18
Like TPM, these independent outlets almost all rely on Google Analytics
to gauge their own traffic from month to month. However, even more
than audience estimates these news outlets want to understand and assess
their impact in the wider journalistic sphere, and use any available tools to
track inbound links or references by more established news outlets.
18 J. Clark & T. van Slyke, Investing in Impact: Media Summits Reveal Pressing
Needs, Tools for Evaluating Public Interest Media, 12 May 2010.
IV. Conclusion & Recommendations: A Role for the Tow Center
The foregoing analysis of the world of Internet metrics points to several
immediate conclusions:
First, the measurement landscape online is far more confused and
uncertain than in traditional media, with no established ``currency'' metric
for comparing audiences. This is generally understood to be a problem by
both publishers and advertisers, though within each group some negotiate
the confusion better than others.
Second, the lack of a single measurement standard is due mainly to two
factors: a natural abundance of data about online audiences, and a
diminished need for a currency metric to plan campaigns and close
advertising deals. Put another way, the prevalence of a clear standard in
television and radio has been a function of a relative dearth of
information.
Third, the prospects for standardization remain uncertain despite various
ongoing efforts to reconcile competing metrics. However, the adoption of
``hybrid'' methodology and the prospect of a merger between comScore
and Nielsen suggest how a dominant audience measurement standard
might emerge.
Fourth, such an audience‐measurement standard, if it emerges, will not
play the decisive role of traditional media currencies such as Nielsen’s TV
ratings. The abundance of information and the diversity of advertising
models strongly suggest that measuring and comparing online media will
remain a complex endeavor involving many data sources.
Fifth, despite the clarion call for clear standards, what news outlets
working online mostly wish for is more and better data: in particular,
detailed information about their own local audience and about their direct
competitors. What the best‐placed online publishers possess is not clarity,
but control — control of the detailed demographic and behavioral
information which advertisers can seek from an ever‐widening array of
sources.
This initial analysis suggests three broad avenues for the Tow Center:
First, educating journalists and journalism students to understand and
monitor the Internet measurement industry; second, cultivating data to
track and assess business models in online news; and third, developing
approaches to think about and measure journalistic success in the
networked public sphere.
Broadening the scope
The challenge of Internet audience measurement is just one part of a
larger shift in the media landscape, brought about by a rapid evolution of
communications technology. These changes are ongoing — innovations in
mobile devices, tablet computing, and other new platforms and content
models may challenge the dominance of content delivered by Web
browsers on personal computers. Furthermore, the growing importance of
social media as the navigational frame that brings eyeballs to content will
alter dynamics of exposure and impact in unforeseeable ways. Media
industries will face an evolving challenge to their business models, and
better solutions to today’s problems may not have a long shelf life.
The news media face an indefinite period of uncertainty, with few anchors
to hold old business models in place. Furthermore, advertising faces a
challenge larger than just changes in the way people consume
professionally produced media content. The Internet collapses the social
and communicative distance between all parties: individuals, businesses,
government, civil society, celebrities, subcultures, and knowledge elites,
etc. For marketers, camping their signal onto professional media of any
kind is no longer the obvious primary strategy. As Clay Shirky put it in
his much‐cited essay on the Internet’s challenge to newspapers, ``That the
relationship between advertisers, publishers, and journalists has been
ratified by a century of cultural practice doesn’t make it any less
accidental.''
To paraphrase Doc Searls, the fact that all parties are now one degree
away from each other changes everything. In fact, as this report has
documented, improvements in online measurement may harm rather than
help advertising as traditionally understood. As Searls says, ``Advertising
amounts to buying guesswork, and technology will reduce the guesswork
in ways that will not be friendly to the business. The frontier of the media
industry is not advertisement.''
Whether Searls’s claim will hold is far from certain, but the scope of the
uncertainty itself is the point. Confused audience measurement is just part
of a larger problem facing professional media, and journalism in
particular. In contemplating a role for Columbia’s new Tow Center, it is
important to bring a wider landscape into the picture. How might an
academic center make a meaningful contribution to the future of
journalism, one that embraces advanced Internet data and metrics without
being constrained by them?
A good way to start is by rethinking the question: How can academic
research contribute to preserving and improving the practice of
journalism as the institutional foundations of the profession shift
unpredictably in the face of rapidly evolving technologies and consumer
behavior? Analysis of Internet data can be enormously valuable, used in
the service of examining the right questions.
A focus on accurate audience estimates and better behavior metrics draws
the eye away from more basic questions of impact which capture the
social value of journalism: Did this story spread? How has it made a
difference? Does the work of reporters lead to more public accountability?
Just what, actually, is the emerging role professional journalism in what
The news profession and the society it serves need to know how
journalistic outputs affect public affairs and political processes in a time
when Internet data show so powerfully what survey research has always
suggested: that the distribution of knowledge and attention to public
affairs is extremely unequal. The current crisis in commercial media is an
opportunity for the profession to reexamine old assumptions about the
relationship between the press, the public, and democracy. It is not
enough for journalists and editors to fall back on abstract moral claims
about the profession’s public role, while telling the business side of their
organizations to get savvy about metrics for the sake of saving old
business models.
Jay Rosen, professor of Journalism at New York University, says this
about the collision of established forms of journalism with new kinds of
19 Y. Benkler, The Wealth of Networks, Yale University Press, 2006:
chapter 7. (Accessible at
http://www.benkler.org/Benkler_Wealth_Of_Networks_Chapter_7.pdf).
data: ``The difficulty with Internet data is that the reaction to it is one of
two extremes: journalists either ignore it completely, or become slaves to
it. Neither is a rational position for 21st Century journalists.''
An opportunity exists for the Tow Center to help the profession move past
this first‐generation reaction of denial or submission, toward a more
mature position of knowledge. Sophisticated empirical analysis can help
understand the role of journalism in the evolving media ecosystem, in
ways that better equip journalists to make a difference, get credit for the
difference they are making, and find ways to get paid, whether through
old or new models. Let us consider three approaches in turn.
1. Understanding measurement data
A clear role for the Tow Center echoed by academic experts as well as
publishers lies in educating journalists to navigate the chaos of data about
online audiences, and in particular about journalism on the Internet. The
NAA’s Randy Bennett argues that the most critical contribution the Tow
Center can make is to enhance journalists’ generally thin understanding of
the measurement and ratings world.
Poorly imagined, of course, this reflex risks becoming a non‐response.
One obvious and immediate goal might be to support journalism which
exposes the contentiousness of Internet metrics and monitors ongoing
reforms — for instance, the methodologically opaque ``hybrid'' measures
now being unveiled. The broader objective, however, must be to help
journalists to use data to think critically about the evolving journalistic
landscape itself.
``The question is how journalists can use the numbers to improve
journalism by seeing how people use it,'' Rosen explains, adding that the
goal should be ``to become data‐savvy enough to find where short‐term
user interests and longer term (professional/societal) interests overlap.''
When journalists use Internet data today, it tends to be in unsophisticated
ways aimed at improving fairly simple metrics. The opportunity exists to
marry data on traffic, citations, users, and text to generate sophisticated
insight into how stories spread, agendas are set, and information on key
topics is discovered and used by key audiences.
2. Measuring journalistic influence
In sharp contrast to the deluge of data about the size and shape of online
audiences, there is an almost complete lack of thoughtful measures of how
news stories travel and what effect they have. Journalists themselves care
deeply about how their work fares in the so‐called ``link economy'' — who
is citing, linking to, and talking about the stories they produce. This
requires closer attention to the journalistic ecosystem itself, to the
relationships between and among different kinds of news outlets and
journalists.
The Tow Center can play an important role in developing tools and
approaches to measure journalistic influence in a rapidly changing news
environment. Rosen suggests a key mission for research is to show how
journalists’ work matters, and to help editorial culture maintain authority
in the face of audience metrics aimed at replicating mass‐media business
logic. ``When you donʹt have great data, you use whatever youʹve got,'' he
says. ``The need is for metrics that arise out of the mission and values of
editorial workers, and can support alternative logics for better decisions.''
3. Assessing new business models for news
The academic experts we interviewed each suggested that research on
emerging business models for professional journalism would make a
valuable contribution. Academic research should, in Searls’ words, ``study
the economics of whatever future media might be.'' The Tow Center
should be positioned to make a contribution to journalism both where adsupported
media thrive, and where they do not.
Of course, access to intimate business data from publishers and news
outlets is essential to develop this role in identifying, understanding, and
teaching new business models for journalism. Based on initial queries,
publishers do not reject out of hand the idea of sharing information on
their investments — human, financial, and otherwise — in the Internet,
and on the success of those investments. The history of similar
``benchmarking'' efforts in journalism (for instance, from the NAA)
suggests that obtaining data is difficult and low response rates are the
norm. Nevertheless, it is clear that news ventures are deeply interested in
such comparative analysis and that an independent, university‐based
research center might be in a good position to aggregate data to provide it.
Conclusion: the opportunity
There is an opportunity for Columbia’s Tow Center to lead by, first,
asking challenging and relevant questions; second, initiating and
participating in collaborative, multi‐method studies that respond to those
questions; and third, disseminating this knowledge within the
community. Yochai Benkler emphasizes the importance of applying
multiple research methods, in combinations appropriate for particular
questions, and in dialogue with similar research organizations and efforts
around the world.
The possibility of collecting rich datasets from publishers and other
industry sources should be considered in the context of building this
research capability. A danger is that sitting on a stockpile of data would,
over time, bias the questions asked to fit the data available. But there is an
enormous opportunity to use a high value archive of otherwise
unavailable data as the centerpiece of a strategy to attract collaborators
and research funding. Early work that successfully mines insights from
novel data would attract follow‐on work and spark new research ideas in
the broader academic and commercial communities. The Center could
thus participate in studies conceived externally as well as internally.
Additionally, rich data could be a useful resource for the joint
journalism/computer science degree program. Computer science students
could contribute a lot to the mining of Internet data, as well as provide a
good deal of the expertise that would be required to store, access and
analyze it.
It is feasible to build a strong research component into the Tow Center’s
activities, and empirical work on Internet data can certainly build
knowledge that will help journalists understand how their work finds
traction in the networked public sphere. Rosen suggests that the culture of
journalism suffers from an anti‐technology, anti‐data, anti‐business bias
that has ``infantilized'' the profession in face of the Internet’s challenge. A
valuable mission for the Tow Center is to overcome this bias with
empirical research which arms journalists and editors with the knowledge
to do their own jobs better, letting them take responsibility for using new
media tools and Internet data to further the profession’s objectives. This
review of the chaotic landscape of online measurement suggests one good,
informal mission statement: To ensure that the responsibility for
understanding and using Internet metrics in journalism does not rest
solely with people selling ads.
Funding for this research was provided by Mary Graham, a member of
the Board of Visitors of the Columbia University Graduate School of
Journalism.
Author Biographies:
Lucas Graves is a PhD candidate in Communications at Columbia
University. His dissertation research uses network analysis to study
structural changes in the current‐affairs ecosystem. Lucas has worked as a
technology and media analyst with Digital Technology Consulting and
before that with Jupiter Research, where he covered Web technology as
well as online advertising and commerce. He is also is a longtime
magazine journalist, now on the masthead of Wired magazine. Lucas holds
an MA in Communications and an MS in Journalism from Columbiaʹs JSchool.
John Kelly is a graduate of the PhD program in Communications at
Columbia University and has studied communications at Stanford and at
Oxford’s Internet Institute. His research blends Social Network Analysis,
content analysis, and statistics to solve the problem of making complex
online networks visible and understandable. John is the founder and lead
scientist of Morningside Analytics, and previously served as Principal
Investigator of Columbiaʹs Interactive Design Lab. He is an Affiliate at the
Berkman Center for Internet and Society at Harvard Law School.
Marissa Gluck is a writer, speaker and consultant covering the marketing
and media industries. Marissa is founder and managing partner of Radar
Research; formerly she worked as a senior analyst at Jupiter Research and
as emerging technologies specialist at i‐traffic. She has advised clients
including Interpublic Group, WPP, DoubleClick, Nielsen NetRatings,
Google, CBS, AOL Time Warner, and the Norman Lear Center. Marissa
holds masters degrees in Communications from the London School of
Economics and from USCʹs Annenberg School.